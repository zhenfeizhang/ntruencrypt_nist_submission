%\documentclass[10pt]{acmlarge}
%\documentclass[10pt, onepage]{IEEEtran}
%\documentclass[10pt]{llncs}
\documentclass{llncs}
\usepackage{amssymb,amscd,amsmath,url,hyperref}
\usepackage{algorithmic}
\usepackage{algorithm}
\usepackage{graphicx}
\usepackage{enumitem}
\usepackage{bytefield}
%%%%%%%%%%%%%%%%%%%%%%%%%%%%%%%%%%%%%%%%%%%%%%%%%%%%%%%%%%%%%%%%%%%%%%

\renewcommand{\algorithmicrequire}{\textbf{Input:}}
\renewcommand{\algorithmicensure}{\textbf{Output:}}
\renewcommand{\qed}{\hfill\blacksquare}
%\allowdisplaybreaks

% Theorem environments

%\newtheorem{theorem}{Theorem}
%\newtheorem{lemma}[theorem]{Lemma}
%\newtheorem{conjecture}[theorem]{Conjecture}
%\newtheorem{question}[theorem]{Question}
%\newtheorem{proposition}[theorem]{Proposition}
%\newtheorem{corollary}[theorem]{Corollary}

%\theoremstyle{definition}
% The * surpresses numbering
%\newtheorem*{definition}{Definition}
\newtheorem{heuristic}{Heuristic}
\newtheorem{assumption}{Assumption}

%\theoremstyle{remark}
%\newtheorem{remark}[theorem]{Remark}
%\newtheorem*{acknowledgement}{Acknowledgements}

%\newcommand\claim[2]{\par\vspace{1ex minus 0.5ex}\noindent%
%\textbf{Claim #1}.\enspace\emph{#2}.\par\noindent\ignorespaces}

%\numberwithin{equation}{section}

%%%%%%%%%%%%%%%%%%%%%%%%%%%%%%%%%%%%%%%%%%%%%%%%%%%%%%%%%%%%%%%%%%%%%%

%%%%%%%% Set Up Environment for Parts in Theorems %%%%%%%%%%%%%%
\newenvironment{parts}[0]{%
  \begin{list}{}%
    {\setlength{\itemindent}{0pt}
     \setlength{\labelwidth}{1.5\parindent}
     \setlength{\labelsep}{.5\parindent}
     \setlength{\leftmargin}{2\parindent}
     \setlength{\itemsep}{0pt}
     }%
   }%
  {\end{list}}
% Use \Part{(a)}, instead of \item[(a)], to ensure upright font
\newcommand{\Part}[1]{\item[\upshape#1]}

%%%%%%%%%%%%%%%%%%
% Greek Alphabet %
%%%%%%%%%%%%%%%%%%
\renewcommand{\a}{\alpha}
\renewcommand{\b}{\beta}
\newcommand{\g}{\gamma}
\renewcommand{\d}{\delta}
\newcommand{\e}{\epsilon}
\newcommand{\f}{\phi}
\renewcommand{\l}{\lambda}
\renewcommand{\k}{\kappa}
\newcommand{\lhat}{\hat\lambda}
\newcommand{\m}{\mu}
\newcommand{\bfmu}{{\boldsymbol{\mu}}}
\renewcommand{\o}{\omega}
\renewcommand{\r}{\rho}
\newcommand{\rbar}{{\bar\rho}}
\newcommand{\s}{\sigma}
\newcommand{\sbar}{{\bar\sigma}}
\renewcommand{\t}{\tau}
\newcommand{\z}{\zeta}

\newcommand{\D}{\Delta}
\newcommand{\G}{\Gamma}
\newcommand{\F}{\Phi}

%%%%%%%%%%%%%%%%%%%%
% Fraktur Alphabet %
%%%%%%%%%%%%%%%%%%%%
\newcommand{\ga}{{\mathfrak{a}}}
\newcommand{\gb}{{\mathfrak{b}}}
\newcommand{\gn}{{\mathfrak{n}}}
\newcommand{\gp}{{\mathfrak{p}}}
\newcommand{\gP}{{\mathfrak{P}}}
\newcommand{\gq}{{\mathfrak{q}}}

%%%%%%%%%%%%%%%%%%%
% Barred Alphabet %
%%%%%%%%%%%%%%%%%%%
\newcommand{\Abar}{{\bar A}}
\newcommand{\Ebar}{{\bar E}}
\newcommand{\Pbar}{{\bar P}}
\newcommand{\Sbar}{{\bar S}}
\newcommand{\Tbar}{{\bar T}}
\newcommand{\ybar}{{\bar y}}
\newcommand{\phibar}{{\bar\phi}}

%%%%%%%%%%%%%%%%%%%%%%%%%
% Calligraphic Alphabet %
%%%%%%%%%%%%%%%%%%%%%%%%%
\newcommand{\Acal}{{\mathcal A}}
\newcommand{\Bcal}{{\mathcal B}}
\newcommand{\Ccal}{{\mathcal C}}
\newcommand{\Dcal}{{\mathcal D}}
\newcommand{\Ecal}{{\mathcal E}}
\newcommand{\Fcal}{{\mathcal F}}
\newcommand{\Gcal}{{\mathcal G}}
\newcommand{\Hcal}{{\mathcal H}}
\newcommand{\Ical}{{\mathcal I}}
\newcommand{\Jcal}{{\mathcal J}}
\newcommand{\Kcal}{{\mathcal K}}
\newcommand{\Lcal}{{\mathcal L}}
\newcommand{\Mcal}{{\mathcal M}}
\newcommand{\Ncal}{{\mathcal N}}
\newcommand{\Ocal}{{\mathcal O}}
\newcommand{\Pcal}{{\mathcal P}}
\newcommand{\Qcal}{{\mathcal Q}}
\newcommand{\Rcal}{{\mathcal R}}
\newcommand{\Scal}{{\mathcal S}}
\newcommand{\Tcal}{{\mathcal T}}
\newcommand{\Ucal}{{\mathcal U}}
\newcommand{\Vcal}{{\mathcal V}}
\newcommand{\Wcal}{{\mathcal W}}
\newcommand{\Xcal}{{\mathcal X}}
\newcommand{\Ycal}{{\mathcal Y}}
\newcommand{\Zcal}{{\mathcal Z}}

%%%%%%%%%%%%%%%%%%%%%%%%%%%%
% Blackboard Bold Alphabet %
%%%%%%%%%%%%%%%%%%%%%%%%%%%%
\renewcommand{\AA}{\mathbb{A}}
\newcommand{\BB}{\mathbb{B}}
\newcommand{\CC}{\mathbb{C}}
\newcommand{\FF}{\mathbb{F}}
\newcommand{\GG}{\mathbb{G}}
\newcommand{\PP}{\mathbb{P}}
\newcommand{\QQ}{\mathbb{Q}}
\newcommand{\RR}{\mathbb{R}}
\newcommand{\ZZ}{\mathbb{Z}}

%%%%%%%%%%%%%%%%%%%%%%%%%%
% Boldface Math Alphabet %
%%%%%%%%%%%%%%%%%%%%%%%%%%
\newcommand{\bfa}{{\bf a}}
\newcommand{\bfb}{{\bf b}}
\newcommand{\bfc}{{\bf c}}
\newcommand{\bfe}{{\bf e}}
\newcommand{\bfep}{{\bf \epsilon}}
\newcommand{\bff}{{\bf f}}
\newcommand{\bfg}{{\bf g}}
\newcommand{\bfh}{{\bf h}}
\newcommand{\bfm}{{\bf m}}
\newcommand{\bfn}{{\bf n}}
\newcommand{\bfp}{{\bf p}}
\newcommand{\bfr}{{\bf r}}
\newcommand{\bfs}{{\bf s}}
\newcommand{\bft}{{\bf t}}
\newcommand{\bfu}{{\bf u}}
\newcommand{\bfv}{{\bf v}}
\newcommand{\bfw}{{\bf w}}
\newcommand{\bfW}{{\bf W}}
\newcommand{\bfx}{{\bf x}}
\newcommand{\bfy}{{\bf y}}
\newcommand{\bfz}{{\bf z}}
\newcommand{\bfA}{{\bf A}}
\newcommand{\bfF}{{\bf F}}
\newcommand{\bfB}{{\bf B}}
\newcommand{\bfC}{{\bf C}}
\newcommand{\bfG}{{\bf G}}
\newcommand{\bfH}{{\bf H}}
\newcommand{\bfI}{{\bf I}}
\newcommand{\bfM}{{\bf M}}
\newcommand{\bfP}{{\bf P}}
\newcommand{\bfS}{{\bf S}}
\newcommand{\bfT}{{\bf T}}
\newcommand{\bfzero}{{\bf{0}}}
\newcommand{\bfU}{{\bf U}}
\newcommand{\bfV}{{\bf V}}
\newcommand{\pqntrusign}{{\sf{pqNTRUSign}}}
\newcommand{\ntru}{{\sf{NTRU}}}
\newcommand{\ntrukem}{{\sf{ntru-kem}}}
\newcommand{\ntrupke}{{\sf{ntru-pke}}}
\newcommand{\ssntrukem}{{\sf{ss-ntru-kem}}}
\newcommand{\ssntrupke}{{\sf{ss-ntru-pke}}}
\newcommand{\NTRUEncrypt}{{\sf{NTRU}}}
\newcommand{\encap}{{\textsc{Encap}}}
\newcommand{\decap}{{\textsc{Decap}}}
\newcommand{\encrypt}{{\textsc{Encrypt}}}
\newcommand{\decrypt}{{\textsc{Decrypt}}}
\newcommand{\keygen}{{\textsc{KeyGen}}}
\newcommand{\hash}{{\textsc{Hash}}}
%%%%%%%%%%%%%%%%%%%%%%%%%%%%%%
% Miscellaneous New Commands %
%%%%%%%%%%%%%%%%%%%%%%%%%%%%%%
\newcommand{\Aut}{\operatorname{Aut}}
\newcommand{\Disc}{\operatorname{Disc}}
\newcommand{\Div}{\operatorname{Div}}
\newcommand{\End}{\operatorname{End}}
\newcommand{\Gal}{\operatorname{Gal}}
\newcommand{\GL}{\operatorname{GL}}
\newcommand{\Index}{\operatorname{Index}}
\newcommand{\Image}{\operatorname{Image}}
\newcommand{\LCM}{\operatorname{LCM}}
\newcommand{\Lift}{\operatorname{Lift}}
\newcommand{\liftable}{{\textup{liftable}}}
\newcommand{\LS}[2]{{\genfrac{(}{)}{}{}{#1}{#2}}} % Legendre symbol
\newcommand{\tLS}[2]{(#1{}|{}#2)} % Legendre symbol in text (a|b)
\newcommand{\hhat}{{\hat h}}
\newcommand{\Ker}{{\operatorname{ker}}}
\newcommand{\MOD}[1]{~(\textup{mod}~#1)}
\newcommand{\Norm}{{\operatorname{\mathsf{N}}}}
\newcommand{\notdivide}{\nmid}
\newcommand{\normalsubgroup}{\triangleleft}
\newcommand{\odd}{{\operatorname{odd}}}
\newcommand{\onto}{\twoheadrightarrow}
\newcommand{\ord}{\operatorname{ord}}
\newcommand{\Pic}{\operatorname{Pic}}
\newcommand{\Prob}{\operatorname{Prob}}
\newcommand{\Qbar}{{\bar{\QQ}}}
\newcommand{\rank}{\operatorname{rank}}
\newcommand{\Resultant}{\operatorname{Resultant}}
\renewcommand{\setminus}{\smallsetminus}
\newcommand{\Span}{\operatorname{Span}}
\newcommand{\Spec}{\operatorname{Spec}}
\newcommand{\tors}{{\textup{tors}}}
\newcommand{\Trace}{\operatorname{Trace}}
\newcommand{\UHP}{{\mathfrak{h}}}    % Upper half plane
\newcommand{\<}{\langle}
\renewcommand{\>}{\rangle}

\newcommand{\longhookrightarrow}{\lhook\joinrel\longrightarrow}
\newcommand{\longonto}{\relbar\joinrel\twoheadrightarrow}
\newcommand{\bens}{\begin{eqnarray}}
\newcommand{\eens}{\end{eqnarray}}

\newcommand*{\defeq}{\mathrel{\vcenter{\baselineskip0.5ex \lineskiplimit0pt
                     \hbox{\scriptsize.}\hbox{\scriptsize.}}}%
                     =}
                     
\newcommand{\LINEFOR}[2]{%
    \STATE\algorithmicfor\ {#1}\ \algorithmicdo\ {#2} \algorithmicend\ \algorithmicfor%
}                     
\newcommand{\LineIf}[2]{ \STATE \algorithmicif\ {#1}\ \algorithmicthen\ {#2} \algorithmicend\ \algorithmicif }
\newcommand{\pluseq}{\mathrel{+}=}        

%%%%%%%%%%%%%%%%%%%%%%%%%%%%%%%%%%%%%%%%%%%%%%%%%%%%%%%%%%%%%%%%%%%%%%



\begin{document}


\title{
	NIST PQ Submission: NTRUEncrypt\\
	A lattice based encryption algorithm
}

\author{Cong Chen \inst{2} \and Jeffrey Hoffstein \inst{1}  \and William Whyte \inst{2} \and Zhenfei Zhang \inst{2} 
}
\institute{Brown University, Providence RI, USA,\ \email{jhoff@math.brown.edu} \and 
OnBoard Security, Wilmington MA, USA,\ \email{\{cchen,wwhyte,zzhang\}@onboardsecurity.com}
}
\date{\today}
\maketitle
%\maketitle
%\begin{abstract}


%\end{abstract}

%\clearpage

%\tableofcontents

%\clearpage

\section{Cover Sheet}
This is an overview document of \ntru~lattice-based cryptosystem for the submission of NIST post-quantum cryptography call for 
standardization.
The submitted cryptosystem consists of : 

\begin{itemize}
\item \ntrupke: a public key encryption (PKE) scheme based on the ``{\em original NTRU}" encryption algorithm by Hoffstein, Pipher and Silverman \cite{DBLP:conf/ants/HoffsteinPS98} with parameter sets derived from a recent revision \cite{DBLP:conf/ctrsa/HoffsteinPSSWZ17}, that achieves CCA-2 security via NAEP transformation \cite{DBLP:journals/iacr/Howgrave-GrahamSSW03};

\item \ntrukem: a key encapsulation scheme (KEM) uses the above public 
key encryption algorithm;


\item \ssntrupke: a public key encryption scheme based on the provable secure  NTRU encryption algorithm \cite{DBLP:conf/eurocrypt/StehleS11} that achieves CCA-2 security via NAEP transformation \cite{DBLP:journals/iacr/Howgrave-GrahamSSW03};


\item \ssntrukem: a key encapsulation scheme (KEM) uses the above public 
key encryption algorithm.
\end{itemize}

\noindent
This documents addresses the following  requirements:

\begin{itemize}
\item {\bf Specifications}
\item {\bf Performance analysis}
\item {\bf A statement of the advantages and limitations}
\item {\bf Cover sheet}
\item {\bf Reference implementation}
\item {\bf Security analysis}
\item {\bf Statement of IPR}
\end{itemize}

\noindent
Submission information:
\begin{itemize}
\item {\bf Principal submitter:} Zhenfei Zhang, zzhang@onboardsecurity.com, Onboard Security, 187 Ballardvale St. Suite A202, Wilmington, MA, 01887, U.S.
\item {\bf Auxiliary submitters:}
Chen Cong, Jeffrey Hoffstein and William Whyte.
\item {\bf 
Inventors of the cryptosystem:} Jeffrey Hoffstein,
               Jill Pipher,
               John M. Schanck,
               Joseph H. Silverman,
               William Whyte and
               Zhenfei Zhang.
\item {\bf Name of the owner of the cryptosystem:} 
Onboard Security Inc.      

\item {\bf Backup point of contact:} William Whyte, wwhyte@onboardsecurity.com, Onboard Security, 187 Ballardvale St. Suite A202, Wilmington, MA, 01887, U.S.
     
\end{itemize}


\section{Algorithm Specifications}
\subsection{Notation}

We use lower case bold letters for vectors, upper case bold letters for matrices.
For a polynomial $f(x) = f_0+f_1x+\dots+ f_{n-1}x^{n-1}$, 
we denote its vector form by $\bff \defeq \left<f_0, f_1, \dots, f_{n-1}\right>$.  We sometimes abuse the notation of vector and polynomial when there is no ambiguity.
For a polynomial/vector $\bff$, the norms are $\|\bff\| \defeq \sqrt{\sum_{i=0}^{n-1}f_i^2}$ and $\|\bff\|_\infty \defeq \max(|f_i|)$. 

We often use the polynomial rings $\Rcal_q \defeq \ZZ[x]/F(x)$ with $F(x) = x^n\pm 1$. %When
%an element of $\ZZ_q$ is lifted to $\ZZ$, or reduced modulo $q$, it is identified with its
%unique representative in $[-q/2,q/2)\cap \ZZ$.
A cyclic rotated matrix of a polynomial $f(x)$ over the ring $\Rcal_q$
is a matrix $\bfM = (\bff_1,\bff_2,\dots,\bff_n)^T$ with $\bff_i = f(x)x^{i-1}\bmod F(x)$.
If  $F(x) = x^n- 1$ it is literally cyclic, and close to cyclic, up to signs, if $F(x) = x^n + 1$.

For a real $a$, we let 
$\lfloor a\rceil$ denote the closet integer to $a$.
 For an integer $a$, we use
$[a]_q$ to denote $a\bmod q$; $\lfloor a\rfloor_p \defeq (a - [a]_p )/p$ for the operation of rounding $a$ to the closest multiple of $p$. 
Modular operations are center lifted, for example $a\bmod q$ returns an integer within $-q/2$ and $q/2$. These notations are also extended to vectors and matrices.

%In the following we will present the scheme over a polynomial ring $\Rcal_q = \ZZ_q[x]/(x^N - 1)$.
%Our scheme also works over other rings, such as $\ZZ_q[x]/(x^N + 1)$ with minor modification.




We set the notation:
\begin{align*}
  \Tcal_N&=\left\{\text{trinary polynomials}\right\}\\
  \Tcal_N(d,e)&=\left\{
        \begin{tabular}{l}
        trinary polynomials with exactly\\ 
        $d$ ones and $e$ minus ones\\
        \end{tabular}
      \right\}
\end{align*}
If~$N$ is fixed we will write~$\Tcal$ and $\Tcal(d,e)$ instead.

%\subsection{The scheme}
\subsection{\ntru~lattice}
Let $f(x)$, $g(x)$ and $h(x)$ be $3$ polynomials in $\Rcal_q$,
where $f(x)$ and $g(x)$ have very small coefficients; $h(x) = p^{-1}g(x)f^{-1}(x)$.  We express by
$\bff$, $\bfg$  and  $\bfh$ the vector form of the polynomials. Also let $\bfF$, $\bfG$ and  $\bfH$ be the matrix  obtained from 
nega-cyclic rotations.
The NTRU lattice with regard to $h$ is defined as 
\[
\Lcal_h = \{(u,v) \in \Rcal_q^2: uh = v\}
\]
or rather, the vector/matrix form:
\[
\Lcal_h = \{(\bfu,\bfv): \bfu\bfH = \bfv \bmod q\}
\]
where there exists a public basis
$\bf P  = \begin{bmatrix}
0& q\bfI_N \\
\bfI_N& \bfH 
\end{bmatrix}
$
and a secret generator
$
[p\bfF|\bfG]
$. %We also require $g(x)$ to be invertible over $\Rcal_p$,
%which is the same as  $\bfG$ being invertible mod $p$.

\subsection{Auxiliary functions}
Let us firstly define some auxiliary functions. Those functions are building blocks for our algorithm.
We give a generic description for those functions.
Implementer may choose to use better (more secure
or more efficient) instantiations when and if they
are available.
Also note that Gaussian samplers are only used by \ssntrupke~and
\ssntrukem.

\paragraph{Hash function.} Through out the paper
we will use \hash~to denote a cryptographic secure hash function that takes arbitrary input length and
outputs a binary string with arbitrary length. In
our submission, we use \textsf{SHA3-512} for such
instantiation.


\paragraph{Trinary Polynomial Generation function.}
This implementation require a function that samples uniformly from 
$\Tcal$ and $\Tcal(d,e)$. This function can be build deterministically
 via a {\em seed} and a hash function.
 

\paragraph{Discrete Gaussian sampler (DGS).} Input a
dimension $N$ and a standard deviation $\sigma$ it outputs a discrete Gaussian distributed vector. In this implementation we use Box-Muller
approach. We remark that there are better samplers in terms of 
efficiency or security. We leave the investigation of those samplers
to future work.

\paragraph{Deterministic Discrete Gaussian sampler (DDGS).}Input a
dimension $N$, a standard deviation $\sigma$ and
a seed $s$, it deterministically outputs a discrete Gaussian distributed vector.


\subsection{The schemes}
We sketch the algorithms related to our proposed schemes. For simplicity
and clearness of the presentation, we omit minor details 
in this high level description. Those includes, for example, checking
the length of the message and encoding/packing ring elements into binary strings.
For those detail we refer the readers to the implantation specification
document submitted also with this document.

\subsubsection{The \ntrupke~scheme}. The \ntrupke~schemes use Algorithms
\ref{alg:ntrukeygen}, \ref{alg:ntruencrypt} and \ref{alg:ntrudecrypt}.

In an \ntru~cryptosystem, $\bff$ (and $\bfg$, if
required) are 
the private keys, while $\bfh$ is the public key.
Those keys can be generated via algorithm
%The key generation algorithm is shown in Algorithm 
\ref{alg:ntrukeygen}.
Note that we use the classical \ntru~flat form (non-product form, cf. \cite{DBLP:journals/iacr/HoffsteinPSSWZ15}) keys 
with a pre-fixed number of $+1$s and $-1$s.

\begin{algorithm}
\caption{{\ntrupke}.\keygen}
\begin{algorithmic}[1]\label{alg:ntrukeygen}
\REQUIRE Parameters $N$, $p$, $q$, $d$ and a {\em seed}.
\STATE Instantiate $\textsf{Sampler}$ with $\mathcal{T}(d+1,d)$ and {\em seed};
\STATE $\bff \gets \textsf{Sampler}$  
\LineIf {$\bff$ is not invertible mod $q$} {go to step 2}
\STATE $\bfg \gets \textsf{Sampler}$
%\LineIf {$\bfg$ is not invertible mod $p$} {go to step 4}
\STATE $\bfh = \bfg/(p\bff+1) \bmod q$
\ENSURE Public key $\bfh$ and secret key $(p\bff,\bfg)$

\end{algorithmic}
\end{algorithm}


\begin{algorithm}[H]
\caption{\ntrupke.\encrypt}
\begin{algorithmic}[1]
\label{alg:ntruencrypt}
\REQUIRE Public key $\bfh$, message $msg$ of length $mlen$, a parameter set and a {\em seed}.

\STATE $\bfm = \text{Pad}(msg, seed)$
\STATE $rseed = \hash(\bfm|\bfh)$
\STATE Instantiate $\textsf{Sampler}$ with $\mathcal{T}$ and {\em rseed};
  \STATE $\bfr \gets \textsf{Sampler}$  
  \STATE $\bft = \bfr*\bfh$
    \STATE $tseed = \hash(\bft)$
   \STATE Instantiate $\textsf{Sampler}$ with $\mathcal{T}$ and {\em tseed}; 
  \STATE $\bfm_{mask} = \textsf{Sampler}$
  \STATE $\bfm' = \bfm - \bfm_{mask} \pmod{p}$
  \STATE $\bfc = \bft + \bfm'$
\ENSURE Ciphertext $\bfc$
\end{algorithmic}
\end{algorithm}

The encryption algorithm in Algorithm \ref{alg:ntruencrypt} uses a padding method to deal with potential
insufficient entropy in a message. The padding algorithm works as follows:
\begin{enumerate}


\item Convert $msg$ into a bit string. This forms the binary coefficients for lower part of polynomial $m$.
\item The last $167$ coefficients of $m(x)$ are 
randomly chosen from $\{-1,0,1\}$. This gives over 
$256$ bits entropy.
\item The length of $msg$ is converted into an $8$ bit binary string, and forms the last $173$ to $168$ coefficients of $m$.    
\end{enumerate}


\begin{algorithm}
\caption{\ntrupke.\decrypt}
\begin{algorithmic}[1]
\label{alg:ntrudecrypt}
\REQUIRE Secret key $\bff$, public key $\bfh$, ciphertext $\bfc$, and a parameter.
  \STATE $\bfm' = \bff*\bfc \pmod{p}$
  \STATE $\bft = \bfc - \bfm$
 \STATE $tseed = \hash(\bft)$
   \STATE Instantiate $\textsf{Sampler}$ with $\mathcal{T}$ and {\em tseed}; 
  \STATE $\bfm_{mask} = \textsf{Sampler}$
  \STATE $\bfm  = \bfm' + \bfm_{mask} \pmod{p}$
\STATE $rseed = \hash(\bfm|\bfh)$
\STATE Instantiate $\textsf{Sampler}$ with $\mathcal{T}$ and {\em rseed};
  \STATE $\bfr \gets \textsf{Sampler}$  
  \STATE $msg, mlen = \text{Extract}(\bfm)$
  \IF{$p\cdot \bfr*\bfh = \bft$}
  \STATE $result = msg, mlen$
  \ELSE
  \STATE $result = \bot$
  \ENDIF
  \ENSURE $result$
\end{algorithmic}
\end{algorithm}
The Extract() operation in Algorithm \ref{alg:ntrudecrypt}  is the inverse of Pad() so we omit the details. It
outputs a message $m$ and its length $mlen$.

\subsubsection{The \ntrukem~algorithms}


We recommend that \ntrukem~to be used for ephemeral
key establishments via the following algorithms.
\ntrukem~uses a same key generation algorithm as 
\ntrupke, namely, Algorithm \ref{alg:ntruencrypt}. Here we present the encapsulation and 
encapsulation algorithms in  Algorithms \ref{alg:ntruencap} and \ref{alg:ntrudecap}.

In a nutshell, the \ntrukem~scheme uses an \ntrupke~scheme to transport
an encapsulated secret, and uses both  this secret and the public key
to derive a shared secret via a secure Key Deviation Function (\textsf{KDF}). 


%We remark that when used in an KEM mode, one should 
%use both $m$ and $pk$ to 
%derive the session key, i.e. $\textsf{KDF}(m, pk, \dots)$.


\begin{algorithm}%[H]
\caption{\ntrukem.\encap}
\begin{algorithmic}[1]
\label{alg:ntruencap}
\REQUIRE Public key $\bfh$, a parameter set \textsc{Param}, and a {\em seed}
  
  \STATE $\text{encaped\_secret}\gets \{0,1\}^{8\times \text{CRYPTO\_BYTES}}$ 
  \STATE $\bfc = $\ntrupke.\encrypt$(\bfh, \text{encaped\_secret}, \text{CRYPTO\_BYTES}, \textsc{Param}, seed)$
  \STATE $ss = \textsf{KDF}(\text{encaped\_secret},\bfh)$.
\ENSURE A ciphertext $\bfc$ and the shared secret $ss$. 
\end{algorithmic}
\end{algorithm}


\begin{algorithm}%[H]
\caption{\ntrukem.\decap}
\begin{algorithmic}[1]
\label{alg:ntrudecap}
\REQUIRE Secret key $f$ and a parameter set \textsc{Param}
  
  \STATE $\text{encaped\_secret} =$\ntrupke.\decrypt$(\bff, \bfh,\bfc, \textsc{Param})$;
  \STATE $ss = \textsf{KDF}(\text{encaped\_secret},\bfh)$.
\ENSURE  The shared secret $ss$.
\end{algorithmic}
\end{algorithm}


 %Also the responder needs to pick an $m$ 
%that has sufficient entropy for the given security
%level. In our implementation we require $32$ bytes
%for $m$, regardless of security level. 
 

\subsubsection{The \ssntrupke~algorithms}
The \ssntrupke~schemes use Algorithms
\ref{alg:ssntrukeygen}, \ref{alg:ssntruencrypt} and \ref{alg:ssntrudecrypt}.

\begin{algorithm}
\caption{{\ssntrupke}.\keygen}
\begin{algorithmic}[1]\label{alg:ssntrukeygen}
\REQUIRE Parameters $N$, $p$, $q$, $\sigma$ and a  {\em seed}
\STATE Instantiate $\textsf{Sampler}$ with $\chi_\sigma^N$ and {\em seed};
\STATE $\bff \gets \textsf{Sampler}$, $\bfg \gets \textsf{Sampler}$;
\STATE $\bfh = \bfg/(p\bff+1) \bmod q$
\ENSURE Public key $\bfh$ and secret key $(p\bff,\bfg)$

\end{algorithmic}
\end{algorithm}

\ssntrupke~uses a similar key generation algorithm as 
\ntrupke. The major difference is that $\bff$ and $\bfg$ are sampled
from Gaussian with deviation $\sigma$, rather than 
from $\mathcal{T}(d,d+1)$. In addition, since \ssntrupke~works over the 
polynomial ring $\ZZ_q[x]/(x^N+1)$ where every element has an inverse,
we are not required to check if $\bff$ and $\bfg$ has an inverse.


\begin{algorithm}%[H]
\caption{\ssntrupke.\encrypt}
\begin{algorithmic}[1]
\label{alg:ssntruencrypt}
\REQUIRE Public key $\bfh$, message $msg$ of length $mlen$, and a parameter set
\STATE $\bfm = \text{Pad}(msg)$
  \STATE $\bfr = \text{DDGS}(\bfm|\bfh)$
  \STATE $\bfe = \text{DGS}$
  \STATE $\bft = p\cdot\bfr*\bfh$
  \STATE $\bfm_{mask} = \text{hash}((\bft\bmod p))$
  \STATE $\bfm' = \bfm - \bfm_{mask} \pmod{p}$
  \STATE $\bfc = \bft + p\cdot \bfe + \bfm'$
\ENSURE Ciphertext $\bfc$
\end{algorithmic}
\end{algorithm}



\begin{algorithm}%[H]
\caption{\ssntrupke.\decrypt}
\begin{algorithmic}[1]
\label{alg:ssntrudecrypt}
\REQUIRE Secret key $f$, public key $h$, ciphertext $c$, and a parameter.
  \STATE $\bfm' = \bff*\bfc \pmod{p}$
  \STATE $\bft = \bfc - \bfm$
  \STATE $\bfm_{mask} = \text{hash}((\bft \bmod p))$
  \STATE $\bfm  = \bfm' + \bfm_{mask} \pmod{p}$
  \STATE $\bfr = \text{DDGS}(\bfm|\bfh)$
  \STATE $\bfe = p^{-1}(\bft - \bfr*\bfh)$  
  \IF{$|\bfe|_\infty \geq \tau\sigma$}
    \STATE $result = \bot$
    \ELSE
  \STATE $result = \text{Extract}(\bfm)$
  \ENDIF
  \ENSURE $result$
\end{algorithmic}
\end{algorithm}

\subsection{The \ssntrukem algorithms}
\begin{algorithm}%[H]
\caption{\ssntrukem.\encap}
\begin{algorithmic}[1]
\label{alg:encrypt}
\REQUIRE Public key $h$, message $msg$ of length $mlen$, and a parameter set
  
  \STATE $m = \text{Pad}(msg)$ 
  \STATE $\bfr \gets \chi_\sigma^N$; $\bfe\gets \chi_\sigma^N$ 

  \STATE $\bfc = p\cdot \bfr*\bfh + p\cdot \bfe + \bfm$
\ENSURE Ciphertext $\bfc$
\end{algorithmic}
\end{algorithm}


\begin{algorithm}%[H]
\caption{\ssntrukem.\decap}
\begin{algorithmic}[1]
\label{alg:encrypt}
\REQUIRE Secret key $f$ and a parameter set
  
  \STATE $\bfm = (p\cdot \bfc*\bff) \bmod p$ 
  \STATE $msg = \text{Extract}(\bfm)$
\ENSURE Ciphertext $c$
\end{algorithmic}
\end{algorithm}




\section{Design Rationale}
\subsection{Hardness assumption}

\subsection{Best known attacks}\label{sec:known_attack_security}




\subsection{Advantages and limitations}
\paragraph{Most scrutinized lattice-based cryptosystem}


\paragraph{\ntru~trapdoor.}
In general lattice based signature offers best performance among quantum-safe solutions, in terms of the combination of signature 
sizes and public key sizes. However, the performance changes greatly 
with how the trapdoor is the constructed. \ntru~trapdoor is in 
general the most efficient one in the literature;  yet it survived
20 years of cryptanalysis, which none other lattice based solution 
has gone through.



\paragraph{Potential application for signature aggregation}. {\bf ZZ: shall we mention it at all?}
\subsection{Performance and implementations}
%\subsection{Performance Analysis}

\paragraph{Optimizations not in this submission package.}
There are two optimizations that we are aware of, that are not 
included in this submission package. Namely
\begin{itemize}
\item AVX2 based optimization for polynomial multiplication \cite{ntrutoc}; this accelerates polynomial multiplication for $2.3$ times.
\end{itemize}


\subsection{Known Answer Test Values}




\section{IPR Statement}

\bibliographystyle{plain}
\bibliography{ntrumls}


\end{document}

